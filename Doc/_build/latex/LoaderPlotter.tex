%% Generated by Sphinx.
\def\sphinxdocclass{report}
\documentclass[letterpaper,10pt,english]{sphinxmanual}
\ifdefined\pdfpxdimen
   \let\sphinxpxdimen\pdfpxdimen\else\newdimen\sphinxpxdimen
\fi \sphinxpxdimen=.75bp\relax

\usepackage[utf8]{inputenc}
\ifdefined\DeclareUnicodeCharacter
 \ifdefined\DeclareUnicodeCharacterAsOptional
  \DeclareUnicodeCharacter{"00A0}{\nobreakspace}
  \DeclareUnicodeCharacter{"2500}{\sphinxunichar{2500}}
  \DeclareUnicodeCharacter{"2502}{\sphinxunichar{2502}}
  \DeclareUnicodeCharacter{"2514}{\sphinxunichar{2514}}
  \DeclareUnicodeCharacter{"251C}{\sphinxunichar{251C}}
  \DeclareUnicodeCharacter{"2572}{\textbackslash}
 \else
  \DeclareUnicodeCharacter{00A0}{\nobreakspace}
  \DeclareUnicodeCharacter{2500}{\sphinxunichar{2500}}
  \DeclareUnicodeCharacter{2502}{\sphinxunichar{2502}}
  \DeclareUnicodeCharacter{2514}{\sphinxunichar{2514}}
  \DeclareUnicodeCharacter{251C}{\sphinxunichar{251C}}
  \DeclareUnicodeCharacter{2572}{\textbackslash}
 \fi
\fi
\usepackage{cmap}
\usepackage[T1]{fontenc}
\usepackage{amsmath,amssymb,amstext}
\usepackage{babel}
\usepackage{times}
\usepackage[Bjarne]{fncychap}
\usepackage[dontkeepoldnames]{sphinx}

\usepackage{geometry}

% Include hyperref last.
\usepackage{hyperref}
% Fix anchor placement for figures with captions.
\usepackage{hypcap}% it must be loaded after hyperref.
% Set up styles of URL: it should be placed after hyperref.
\urlstyle{same}

\addto\captionsenglish{\renewcommand{\figurename}{Fig.}}
\addto\captionsenglish{\renewcommand{\tablename}{Table}}
\addto\captionsenglish{\renewcommand{\literalblockname}{Listing}}

\addto\captionsenglish{\renewcommand{\literalblockcontinuedname}{continued from previous page}}
\addto\captionsenglish{\renewcommand{\literalblockcontinuesname}{continues on next page}}

\addto\extrasenglish{\def\pageautorefname{page}}

\setcounter{tocdepth}{1}



\title{Loader Plotter Documentation}
\date{Dec 13, 2017}
\release{1.0}
\author{A. L. Almeida}
\newcommand{\sphinxlogo}{\vbox{}}
\renewcommand{\releasename}{Release}
\makeindex

\begin{document}

\maketitle
\sphinxtableofcontents
\phantomsection\label{\detokenize{index::doc}}


This is a auto-generated documentation of the Loader-Plotter produced.
The main objective of this loader is to grab the Data from ArrayFire format files and a designated folder. With that and the class requirements, it loads the data to \sphinxstyleemphasis{Numpy} arrays so that it can Plot Surface, Plot 3D and Construct Movies.

Requirements:

To run this Loader one need to have installed various modules:
* \sphinxstyleemphasis{Numpy}
* \sphinxstyleemphasis{ArrayFire}
* \sphinxstyleemphasis{MatPlotLib}
* \sphinxstyleemphasis{ImageIO}
* \sphinxstyleemphasis{DateTime}
\index{meshPlot (class in Loader)}

\begin{fulllineitems}
\phantomsection\label{\detokenize{index:Loader.meshPlot}}\pysiglinewithargsret{\sphinxbfcode{class }\sphinxcode{Loader.}\sphinxbfcode{meshPlot}}{\emph{InPath}, \emph{Plot1D=True}, \emph{PlotSlices=False}, \emph{PlotTemporal=False}, \emph{Mov=False}}{}
Class that contains information about the mesh of the data loaded
\index{load\_envelope() (Loader.meshPlot method)}

\begin{fulllineitems}
\phantomsection\label{\detokenize{index:Loader.meshPlot.load_envelope}}\pysiglinewithargsret{\sphinxbfcode{load\_envelope}}{\emph{filename}}{}
Function to load the file in ArrayFire format and convert it to
an array of NumPy format.

\end{fulllineitems}

\index{load\_parameters() (Loader.meshPlot method)}

\begin{fulllineitems}
\phantomsection\label{\detokenize{index:Loader.meshPlot.load_parameters}}\pysiglinewithargsret{\sphinxbfcode{load\_parameters}}{}{}
Reads the file with parameters of the simulation and returns a set
of data in the following order:
dimentions, time step, number of points, vector limits

\end{fulllineitems}

\index{plot\_sequence() (Loader.meshPlot method)}

\begin{fulllineitems}
\phantomsection\label{\detokenize{index:Loader.meshPlot.plot_sequence}}\pysiglinewithargsret{\sphinxbfcode{plot\_sequence}}{\emph{InPath}, \emph{DoMovie}}{}
Function to plot the set of saved data with the matplotlib
plot functions. This makes the output beautiful!!!

\end{fulllineitems}


\end{fulllineitems}




\renewcommand{\indexname}{Index}
\printindex
\end{document}